
\cvsection{\normalsize{Profiles}}
\begin{itemize}
    \item Leetcode: \href{https://leetcode.com/hydro_lyte/}hydrolyte
    \item Codechef: \href{https://www.codechef.com/users/hydro_ly_te}hydrolyte
    \item Codeforces: \href{https://codeforces.com/profile/hydro_lyte}hydrolyte
    
    
\end{itemize}

\cvsection{\normalsize{Achievements}}
\begin{itemize}
    \item \href{https://techchallenge.in.capgemini.com/winners}{Secured 28th position in Capgemini Coding} challenge among of 5K+ participants worldwide.
    \item \href{https://www.techgig.com/codegladiators2020/finaleleaderboard}{Techgig Code Gladiator 2020 finalist}with Rank of 390 out of 20000 participants
    \item Solved 750+ problems on Leetcode.
    \item \href{https://drive.google.com/file/d/1fKLitzNLsrlvMymT-jfMIkWjSkZXA0Rp/view?usp=sharing}Google Code-in Mentor 2019
\end{itemize}

% \cvachievement{\faTrophy}{}{Received accolades at Atos for Best Performance in team.}
% \cvachievement{\faTrophy}{}{Received Best Debut Award at Atos. }
% %\divider
% \cvachievement{\faInstitution}{}{Won 2nd Consolation Prize for paper presented on Cognitive Radio Networks.}
% %\divider
% \cvachievement{\faGraduationCap}{}{Got Selected in "Exclusive Scholar Program" during undergrad.}
% %\divider
% \cvachievement{\faDollar}{}{Awarded with Narotam Sekhsaria Foundation Scholarship}
%\cvsection{Strengths}

%\cvtag{Hard-working (18/24)} 
%\cvtag{Persuasive}
%\cvtag{Motivator \& Leader}

%\divider\smallskip

%\cvtag{UX}
%\cvtag{Mobile Devices \& Applications}
%\cvtag{Product Management \& Marketing}


%\divider

%\cvevent{B.S.\ in Symbolic Systems}{Stanford University}{Sept 1993 -- June 1997}{}

\cvsection{\normalsize{Projects}}
\cvproject{\href{https://github.com/KoulickS/Path-visualization-}{Path Visualizer Tool - \color{black}PyGame}}
\begin{itemize}
\item Implemented shortest path algorithm using A* search.
\item Also includes, other shortest path-finding algorithms such as Dijkstra's
algorithm and BFS.
\item This visualization tool allows the users to find the optimal shortest
path between two given points in an infinite grid and has been built
using PyGame.
\end{itemize}

\cvproject{\href{https://github.com/KoulickS/Numerical-digit-recognition-using-TDA}{Numerical digit recognition - \color{black} Python, Machine Learning}}
\begin{itemize}
\item Transformed images to point cloud data using TDA Mapper
\item Identified each digit individually using component labelling the
images and using Betti Number to identify void of each digit.
\end{itemize}



\cvsection{\normalsize{Skills}}
C++, JavaScript, Java (Core) \newline
%\divider
Data Structures, Algorithms \newline
%\divider
MySQL, OracleDB, Spring boot \newline

\cvsection{\normalsize{Education}}
\cvevent{\small{B.Tech - Information Technology (2018-2022)}}{\small{Jalpaiguri Government Engineering College}}{& & & & & CGPA: 8.60 }{}



%\item Note: If you’re viewing a soft copy, all blue texts
%are hyperlinks, you can click to view details
